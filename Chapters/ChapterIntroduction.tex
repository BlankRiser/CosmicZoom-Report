\chapter{INTRODUCTION} % Main chapter title
\label{ChapterIntroduction} % For referencing the chapter elsewhere, use \ref{ChapterIntroduction} 



The techonology has evolved rapidly and provided us with various ways to communicate on a global scale and assess vast amopunt of information with a click. These benefits can be utilized by various sectors, and one of them is education which can can greatly be made more efficient by removing limitations of time, space and money. Students could watch a topic being thought any time of the day, anywhere and also maybe for free of cost. With the rise of pandemic, and with the restrictions to the people, the technology to teach people has gathered a lot of attention and all the educational institutions are implementing various ways using these technologies. This is the same for organizing various educational events which help students learn much more than their syllabus and provides them a way to essentially choose their career path. My project also is invovled in developing such a website that is used to educate students with a very interactive user experience, and also allows all the age groups be able to access the website with relative ease.
\\
The purpose of this project was to implement an approach of user experience for a website design, that could highlight all the events conducted in the exhibition that also brought about the vision the client i.e Ajith wanted it to be, and also to develop this using the necessary technologies. While working on this project I mostly 
concentrated on revealing and understanding the concepts of UX design which include usability, visual design and human factors affecting the user experience.
The vision that the client wanted wanted was for the website to look simple and yet elegant and to be accessible on any device without any hiccups with great user experience. With a lot of thinking, wireframing, and prototyping we came up with a design and a story that would be narrated by a host while show-casing the website.
The process of designing and developing was divided into various phases like wireframing, designing, prototyping, data gathering, developing front-end, connecting APIs, and the deploying to an in-house server.   


\section{Problem Formulation}
Since this project was for an online exhibition, the main goal was for it to have a very nice user experience, and also to tell a story from the narrator's point of view during the event.
My aim was to understand all the design aesthetics needed for the project, and for that I needed to clearly undestand the scope of this exhibition as this would help me imagine and approach the design as intended by the narrator of the website(i.e the host of the online exhibition).
The user experience and and the libraries that will be used to complete this project would be a problem as everything would have to be customized as the client would want it to be.

\section{Problem Identification}

Clearly, the problem here would be designing a good user experience that bodes well for people of all ages and provides them with a intriguing experience to enjoy the whole exhibition, along with the narrator. User experience concentrates on how the overall design makes the user to feel. To create not just beautiful but also qualitative and well-worked design is why a user experience design is needed. To achieve positive user feelings during using a website, designers should understand users’ goals, desires, fears, behaviors and ambitions. The problem during software development is that the technical approaches/practices are more popular than user-centric ones. Based on a huge number of surveys conducted by the groups with strong reputation in software production, this is a problem which leads to unsuccessful projects. The reason is the lack of attention to user inputs. In the website design the user experience is identified by not just usability alone. It's also impacted by a lot of design components that UX design covers. It includes usability, utility, design, human factors, accessibility, persuasiveness and others. All these factors while designing also afftect the way that a website has to be developed, because the layout needs to be as accurate to the design as possible.


\section{Problem Statement \& Objectives}
The project is meant to design and develop a website that has a good user experience, that can be used by people of all ages without much effort. It should be responsive and visually pleasing pleasing to all kinds of user. 
This website if for an exhibition that is being converted to an online exhibit and needs a lot of design approaches to be used to make it like so.

\section{Limitations}

There are a few limitations regards to this project alone, as I am the only developer who would also design the user experience of the online exhibit and due to the team not being technaical various terminology issues arise, where I have to summarize what I mean, and also the lack of understanding of the domains of which the exhibit is conducted presents an issue by itself. Technically, there is one limitation I would like to highlight; which is not using a database but rather a google sheet api which is not a good approach, but it was done due to the limitation of team not being being technically adept and also because it would reduce my(developer's) burden to constantly keep updating data. 
