\chapter{INTRODUCTION} % Main chapter title
\label{ChapterIntroduction} % For referencing the chapter elsewhere, use \ref{ChapterIntroduction} 



The report shall not have more than five chapters. The title or heading of these chapters are also fixed except for the Chapter named ‘Actual work’. Students have the flexibility to choose the title for this chapter and its sub chapters. 
\\
Introduction chapter for both BTech and MTech. The chapter shall also have the following four mandatory sub-chapters: 


\section{Problem Formulation}
Under this the reason for choosing the particular problem or title for the project shall be explained along with the thought process that was involved in doing so.
Since this project was for an online exhibition, the main goal was for it to have a very nice user experience, and also to tell a story from the narrator's point of view during the event.
My aim was to understand all the design aesthetics needed for the project, and for that I needed to clearly undestand the scope of this exhibition as this would help me imagine and approach the design as intended by the narrator of the website(i.e the host of the online exhibition)

\section{Problem Identification}
The identified problem shall be formulated in a systematic manner and provide clarity to decide on the problem statement and objectives


\section{Problem Statement \& Objectives}
List of project objectives along with the problem statement shall be provided in this section.


\section{Limitations}
This is an academic research based project and has its own limitations with respect to time and other constraints. Also, there might be other limitations of the project work depicted in the reported that may not be obvious from the title. It is very important to mention those limitations and avoid any misconceptions in the minds of the readers and/or evaluators.
