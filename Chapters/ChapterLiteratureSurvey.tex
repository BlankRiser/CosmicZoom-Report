
\chapter{LITERATURE SURVEY AND REVIEW} % Main chapter title
\label{ChapterLiteratureSurvey} % For referencing the chapter elsewhere, use \ref{Chapter1} 

Before developing the website for an exhibition, a lot of thought has to be put to how the website should look like, the amount of content that it should have, and how would the users be able to interact with various components present in the website. The user experience is a necessity here as ut directly relates to the satisfaction the users feel while browsing through the exhibition website. User experience design (UX) is a set of technologies which increase user satisfaction by improving usability and concepts related to interaction between human users and computers. User experience is a significant aspect in creating different kinds of products and services. The web is one of the most important fields in which a user experience design is applied. UX design is a broad sphere consisting of several components that are its constituents. UX design includes usability, human factors, accessibility and various kinds of design and system performance. 

\section{Literature Collection \& Segregation}
 In the book, "The UX Book: Process and guidelines for ensuring a quality user experience" by [1] Hartson et al, it mentions that the user user experience is the totality of the effect or effects felt by a user as a result of interaction with, and the usage context of, a system device, or product, including the influence of usability, usefulness, and emotional impact during interaction, and
 savoring the memory after interaction. The term “interaction with” is broad and embraces seeing, touching, and thinking about the system or product, including admiring it and its presentation before any physical interaction. Simply said, user experience design is an umbrella term for any kind of activity that provides better experience for
 the user. UX concentrates on how the overall design makes the user to feel. 
 \\
 Before actually focusing on the user experience, we need to know how Online exhibition website have been implemented before and try to learn from them.
 [2] Kalfatovic et al in his book "Creating a winning online exhibition: A guide for libraries, archives, and museums. American Library Association" mentions how important it is to segregate the items to be displayed and the exhibition itself, so as not to confuse the people when there is no exhibition but the item still remain to be showcased. [3] Lester P et al gives us insights on why online exhibitions are an alternative for the exhibitions conducted physically. It is a unique approach that takes in a lot of things into consideration but he says that the educator's would be devoid of the enthusiasm they feel in a physical exhibition. [4] Khoon et al in their article mention that web‐based multimedia systems provide an exciting means and communication channel for information access and sharing, with immense potential in public education.
 \\
 With this we also look at [5] Kelly et al guidelines for designing a good website. It is a list of points that should be considered when designing a web site. Many of these
 are points that should be considered when developing any web site, [6] JJ Garret's book "The elements of user experience: user-centered design for the web and beyond" has vital refrences for web and interaction designers that can be taken over worked over to provide a website with excellent user experience.
 

\section{Critical Review of Literature}

Human-computer interaction is about human behavior and is used to drive system design, and human performance is the measurable outcome in using those systems. One major issue that we would need to address is the issue of accessibility. Accessibility has many forms, and its depth and breadth will be determined by the success of the “who are the clients?” process. Visitors to the site may have sight or hearing impairments and need a text or voice reader. In addition, they may not want to download images because their modem is too slow, use both the mouse and the keyboard..etc.
\\It is likely that we would have to come back to designing and then developing some aspects of the exhibition website that would consider both the accessibility and the human-interactivity. While designing we'll have to look at how useful the website is for them, how much they can learn from it, will it be memorable to them, how effective is our content and the user experience, would they need some features that has not yet been thought of, and how effecient is the website to be able to run any kind of device, with any type of internet connectivity.